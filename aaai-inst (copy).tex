%File: AAAI-inst.tex
% WARNING: If you are not an experienced LaTeX user, AAAI does
% NOT recommend that you use LaTeX to format your paper. No
% support for LaTeX is provided by AAAI, and these instructions
% and the accompanying style files are NOT guaranteed to work.
% If the results you obtain are not in accordance with the 
% specifications you received in your packet (or online), you
% must correct the style files or macro to achieve the correct
% result. 
%
% AAAI CANNOT HELP YOU WITH THIS TASK. 
%
% The instructions herein are provided as a general guide for 
% experienced LaTeX users who would like to use that software
% to format their paper for an AAAI Press proceedings or technical
% report or AAAI working notes. These instructions are generic. 
% Consequently, they do not include specific dates, page charges, and so forth. 
% Please consult your specific written conference instructions for 
% details regarding your submission.
%
% Acknowledgments
% The preparation of the \LaTeX{} and Bib\TeX{} files that
% implement these instructions was supported by 
% Schlumberger Palo Alto Research, AT\&T Bell
% Laboratories, Morgan Kaufmann Publishers, and AAAI Press.
\documentclass[letterpaper]{article}
\usepackage{aaai}
\usepackage{times}
\usepackage{helvet}
\usepackage{courier}
\begin{document}
% The file aaai.sty is the style file for AAAI Press 
% proceedings, working notes, and technical reports.
%
\title{Formatting Your AAAI Paper Using the \LaTeX\ Style Files}
\author{AAAI Press\\
American Association for Artificial Intelligence\\
445 Burgess Drive\\
Menlo Park, California 94025--3496\\
proceedings@aaai.org}
\maketitle
\begin{abstract}
\begin{quote}
Robustness is one of the most significant issues which population-based algorithms face. These algorithms should be able to resist against fluctuating responses through the search process. In this paper, we aim at improving Social fabric-based Cultural algorithms in both belief space and population components regarding robustness. At first, we propose a random neighborhood restructuring mechanism to make the communications between individuals more flexible and let them decide on their exploration or exploitation tendencies through the search process. This method improves the robustness of the search behavior in a self-organized manner. Then, inspired from Statistics, we utilize the concept of Confidence Interval to propose a new Normative knowledge source which is more stable against sudden changes in the values of incoming solutions. Then, the performance of both approaches is measured based on the IEEE-CEC2015 testbed	. Compared to the best results of other algorithms, our proposed methods enhance performance on a diverse landscape of numerical optimization problems.  
\end{quote}
\end{abstract}

\section{Introduction}
Cultural algorithms (CA) were introduced by Reynolds as a type of population-based problem-solving approaches which combine biological evolution with socio-cognitive concepts to give an optimization method based on dual inheritance theory. A large number of CA variants have been proposed in the literature. They address a vast range of problems such as real-valued function optimization, discrete problems, dynamic environments, multi-objective optimization, etc. \newline In CA, evolution happens at two levels: Macro-evolutionary level or belief space and micro-evolutionary level or population space. At the population level any population-based algorithm such as GA, EP, PSO, etc. could be deployed. Belief space extracts a generalized knowledge from individuals experience to bias the search direction at the population level. This process is accomplished through deploying five types of knowledge sources in the belief space. Here, each individual in the population space is controlled by a knowledge source. \newline 

Social Fabric is the notion of an infrastructure in which knowledge sources in the belief space can access the social networks in which individuals interact. Knowledge sources are allowed to propagete their influence on the individuals through a hierarchy of layered networks. Each individual might exist in several networks. So, some individuals can play the role of mediator between different networks. \newline
Individuals could be connected in a variety of different topologies. These topologies determine the extent to which the influence of a knowledge source could be propagated through the network. Dense topologies help the individuals to follow (exploit) elite members efficiently while sparse topologies are suitable for exploratory communities. To make the propagation of knowledge between individuals more efficient some neighborhood restructuring schemes are considered by the current extension of CA. Current topologies of networks may transform between different structures to increase/decrease their connectivity rate in the case of stagnation. \newline
Particle Swarm Optimization could be considered as another population-based algorithm which relies on social structures. PSO simulates the social behavior observed in natural systems such as birds flocking, herds, schools, etc. PSO finds the best solution by adjusting the moving vector of each individual (particle) according to its best history and its neighborhood best positions of particles in the entire swarm at each iteration. The neighborhood is determined by the topology of the swarm. For example, dense topologies lead to more interactions and faster convergence. Here, we utilize PSO algorithm at the population level of CA.\newline
In this paper, we aim at improving the robustness of CAs at the both levels of belief space and population space. In the belief space, the standard implementation of Normative ranges will be replaced with the Confidence Interval concept. This approach makes the Normative knowledge source robust against sudden changes in input data. In the population space, the standard tactical neighborhood restructuring is enhanced to a strategy called Random neighborhood restructuring which happens at the particles level. It aims at improving robustness regarding self-organization aspect. 

\section{Related Works}
Many variants of CAs have been proposed in a vast range of different applications such as single and multi-objective optimization, dynamic problems, social interactions simulation, etc. Here, we are interested in studying Social Fabric phenomena as a modern variant of CAs which aims at function optimization. \newline
[]replaced the traditional idea of the roulette wheel with the vector voting model to determine the controller knowledge source of individuals in each iteration. They investigated the effect of different homogeneous topologies in a social fabric. They measured this approach on both static and dynamic problems. Their results confirm that there is no best topology which works well for all problems. So, they concluded that heterogeneous topologies should be considered to make the algorithm scalable for complex problems.\newline
[] divided the whole population into multiple sub-populations (tribes) to construct a two-layer network between individuals. As a diversity-preserving measure, they introduced Tactical Neighborhood Restructuring to avoid stagnation in non-linear multi-modal problems. This strategy facilitate the dissemination of knowledge through the network to help the heterogeneous tribes to prevent local optima.\newline
[]  introduced Tribe-PSO algorithm inspired by Hierarchical Fiar Competition. They divided the population into some tribes to draw a two-layer model which splits the individuals into two layers: elite and rudimentary. The process of convergence is divided into three phases to ensure a reasonable level of diversity. They evaluated their algorithm on De Jong's test functions. As an application, they used their approach for molecular docking purpose for a test set of 100 protein-ligand complexes.
\newline
[] investigates different types of heterogeneity in PSO algorithm in four categories: neighborhood, update rule, model of influence and parameters. Here, heterogentiy means particles might have follow different rules or have different initial parameters or neighborhood sizes. It helps PSO variants to hold enough diversity and prevent early convergence.\newline
[] gives a detailed description of Social Fabric based Cultural Algorithms with neighborhood restructuring. The idea was to investigate the effect of changing neighborhood of sub-populations (tribes) and how it influences the propagation of information through the fabric (network). Communication between tribes happen through the elites of each tribe. In fact, they are mediators between independent tribes. 

-------------------------------------

In the past, this instruction file has contained general formatting instructions
for persons preparing their papers for an AAAI proceedings. Those instructions
are now available on AAAI's website in PDF format only, and are part of the
general author kit distributed to all authors whose papers have been accepted by
AAAI for publication in an AAAI conference proceedings. This document only
contains specific information of interest to \LaTeX\ users. 

{\bf Warning:} If you are not an experienced \LaTeX\ user, AAAI does {\bf not}
recommend that you use \LaTeX\ to format your paper. No support for LaTeX is
provided by AAAI, and these instructions and the accompanying style files are {\bf
not} guaranteed to work. If the results you obtain are not in accordance with the 
specifications you received in your packet (or online), you must correct the style
files or macro to achieve the correct result. AAAI {\bf cannot} help you with this
task.  The instructions herein are provided as a general guide for experienced
\LaTeX\ users who would like to use that software to format their paper for an
AAAI Press proceedings or technical report or AAAI working notes. These
instructions are generic. Consequently, they do not include specific dates, page
charges, and so forth. Please consult your specific written conference
instructions for details regarding your submission.

\section{Output}
To ready your paper for publication, please read the ``Instructions for Authors''
paper. This document is available on AAAI's website, and is supplied in your
author kit.

\subsection{Paper Size}
If you are outside the US, the \LaTeX\ default paper size has most likely been
changed from ``letterpaper" to ``a4." Because we require that your electronic
paper be formatted in US letter size, you will need to change the default back to
US letter size. Assuming you are using the 2e version of \LaTeX\, you can do this
by including the [letterpaper] option at the beginning of your file:
\begin{footnotesize}
\begin{verbatim}
  \documentclass[letterpaper]{article}. 
\end{verbatim}
\end{footnotesize}

This command is usually sufficient to change the format for LaTeX. However,  it
is also usually necessary to configure dvips. Try passing the -tletter option to
dvips. Those using RedHat Linux 8.0 and LaTeX should also check the paper size
setting in  "/usr/share/texmf/dvips/config/config.ps" --- it may be that ``A4" is
the default, rather than ``letter."  This  can result in incorrect top and bottom
margins in documents you prepare with LaTeX. You will need to edit the config
file to correct the problem. (Once you've edited to config file for US letter, it
may not be possible for you to print your papers locally).


\section{The AAAI Style File}
The latest version of the AAAI style file is available on AAAI's website. You
should download this file and place it in a file named ``aaai.sty" in the \TeX\
search path. Placing it in the same directory as the paper should also work. (We
recommend that you download the complete author kit so that you will have the
latest bug list and instruction set.

\subsection{Using the Style File}
In the \LaTeX\ source for your paper, place the following lines as follows:

\begin{footnotesize}
\begin{verbatim}
\documentclass[letterpaper]{article}
\usepackage{aaai}
\usepackage{times}
\usepackage{helvet}
\usepackage{courier}\title{Title}
\author{Author 1 \and Author 2 \\ 
Address line \\ Address line 
\And
Author 3 \\ Address line \\ Address line}
\begin{document}
\maketitle
...
\end{document}
\end{verbatim}
\end{footnotesize}

This command set-up is for three authors. Add or subtract author and address
lines as necessary.  In most instances, this is all you need to do to format your
paper in the Times font. The helvet package will cause Helvetica to be used for
sans serif, and the courier package will cause Courier to be used for the
typewriter font. These files are part of the PSNFSS2e package, which is freely
available from many Internet sites (and is often part of a standard installation.
If using these commands does not work for you (and you are using \LaTeX2e), you
will need to refer to the fonts information found later on in this document.


\subsubsection{Including a Reference List}

At the end of your paper, you can include your reference list by using the
following commands:

\begin{footnotesize}
\begin{verbatim}
\bibliography{Bibliography-File}
\bibliographystyle{aaai}
\end{document}
\end{verbatim}
\end{footnotesize}

\subsubsection{Formatting Author Information}
Author information can be set in a number of different styles, depending on the
number of  authors and the number of affiliations you need to display. For
several authors from the same institution, use \verb+\+and:

\begin{footnotesize}
\begin{verbatim}
\author{Author 1 \and 
... 
\and Author n \\
Address line \\ 
... 
\\ Address line}
\end{verbatim}
\end{footnotesize}

\noindent If the names do not fit well on one line use:

\begin{footnotesize}
\begin{verbatim}
\author{Author 1 \\ {\bf Author 2} \\ 
... 
\\ {\bf Author n} \\
Address line \\ 
... 
\\ Address line}
\end{verbatim}
\end{footnotesize}

\noindent For authors from different institutions, use \verb+\+And:

\begin{footnotesize}
\begin{verbatim}
\author{Author 1 \\ Address line \\
... 
\\ Address line
\And ... \And
Author n \\ Address line \\
... 
\\ Address line}
\end{verbatim}
\end{footnotesize}

\noindent To start a separate ``row" of authors, use \verb+\+AND:
\begin{footnotesize}
\begin{verbatim}
\author{Author 1 \\ Address line \\
... 
\\ Address line
\AND
Author 2 \\ Address line \\
... 
\\ Address line \And
Author 3 \\ Address line \\
... 
\\ Address line}
\end{verbatim}
\end{footnotesize}

\noindent If the title and author information does not fit in the area
allocated, place
\begin{footnotesize}
\begin{verbatim}
\setlength\titlebox{\emph{height}}
\end{verbatim}
\end{footnotesize}
after the \verb+\+documentclass line where \emph{height} is
something like 2.5in.

\subsubsection{Adding Acknowledgements}
To acknowledge other contributors, grant support, or whatever, use
\verb+\+thanks in either the \verb+\+author or \verb+\+title commands.
For example:
\begin{footnotesize}
\begin{verbatim}
\title{Very Important Results in 
AI\thanks{This work is supported by 
everybody.}}
\end{verbatim}
\end{footnotesize}

Multiple \verb+\+thanks commands can be given. Each will result in a
separate footnote indication in the author or title with the
corresponding text at the bottom of the first column of the document.
For example:
\begin{footnotesize}
\begin{verbatim}
\author{A. Researcher\thanks{Now at 
Microsoft.} \andB. Researcher\thanks{Not 
at Microsoft.}}
\end{verbatim}
\end{footnotesize}

One common error with \verb+\+thanks is forgetting to use
\verb+\+protect on what \LaTeX\ calls ``fragile'' commands.

\subsubsection{Adding a Publication Note}
To add a comment to the header of document, use \verb+\+pubnote, as in
\verb+\+pubnote\{\verb+\+emph\{To appear,  AI Journal\}\}
This should be placed after the title and author information but
before \verb+\+maketitle. Note that \verb+\+pubnote is for printing
the paper yourself, and should not be used in submitted versions!

\subsubsection{Copyright information}
By default, the AAAI copyright slug will be printed at the bottom of
the first column of your document. To suppress the copyright slug,
use \verb+\+nocopyright somewhere before \verb+\+maketitle. To change
the year in the copyright slug from the current year, use
\verb+\+copyrightyear\{\emph{year}\}
To change the entire text of the copyright slug, use
\verb+\+copyrighttext\{\emph{text}\}
Either of these must appear before \verb+\+maketitle.

\section{Bibliography Style and References}
The aaai.sty file includes a set of definitions for use in
formatting references with BibTeX. These definitions make the
bibliography style closer to the one specified in the ``Instructions
to Authors'' for AAAI papers.

To use these definitions, you also need the BibTeX style file
aaai.bst, available from the AAAI web site.
Then, at the end of your paper but before \verb+\+end{document}, you
need to put the following lines:

\begin{footnotesize}
\begin{verbatim}
\bibliographystyle{aaai}
\bibliography{bibfile1,bibfile2,...}
\end{verbatim}
\end{footnotesize}

The list of files in the bibliography command should be the
names of your BibTeX source files (that is, the .bib files
referenced in your paper).

The following commands are available for your use in citing
references:
\begin{description}
\item \verb+\+cite: Cites the given reference(s) with a full citation.
This appears as ``(Author Year)'' for one reference, or ``(Author Year;
Author Year)'' for multiple references.
\item \verb+\+shortcite: Cites the given reference(s) with just the year.
This appears as ``(Year)'' for one reference, or ``(Year; Year)''
for multiple references.
\item \verb+\+citeauthor: Cites the given reference(s) with just the
author name(s) and no parentheses.
\item \verb+\+citeyear: Cites the given reference(s) with just the
fate(s) and no parentheses.
\end{description}


\section{Copyright}
If you were required to transfer copyright of your paper to AAAI, you must include
the AAAI copyright notice and web site address on all copies of your paper,
whether electronic or paper (including the camera copy you provide to AAAI.) If
you use the  latest AAAI style file, the copyright line will be inserted for you
automatically. (If your paper doesn't include the copyright slug and it should,
the paper will be returned to you for reformatting.) If we did not require you to
transfer copyright, you may disable the copyright line using the
\verb+\+nocopyrightcommand.

\section{Fonts}
Papers published in AAAI publicatons must now be formatted using the Times family of
fonts, so that all papers in the proceedings have a uniform appearance. If you've
been using Computer Modern, the first advantage you will see to using Times is that
the character count is smaller --- that means you can put more words on a page!

\subsection{Type 3 Fonts}

You've probably seen PDF files containing type 3 bitmapped fonts on the web (there
are files like that on AAAI's web site too). They're often the huge files that open
slowly, scroll very slowly, and aren't readable unless they're enlarged many times. 
Aside from these problems, these files usually contain fonts designed for 300 dpi
printers, so they print at 300 dpi even if the printer resolution is 1200 dpi or
higher. They also often cause high resolution imagesetter devices to crash and 
sometimes the PDF indexing software we use can't read them.

Because of these and other problems, as well as for purposes of uniformity, {\bf AAAI
will not longer accept electronic files where the text and headings are formatted
using obsolete type 3 fonts.} Documents with Type 3 bitmap fonts are not acceptable and will be returned to the authors. 

Fortunately, there are effective workarounds that will
prevent your file from embedding type 3 bitmapped fonts.

The easiest workaround is to use the times, helvet, and courier packages with
\LaTeX\ 2e. (Note that papers formatted in this way will still use Computer Modern
for the mathematics. To make the math look good, you'll either have to use Symbol
or Lucida, or you will need to install type 1 Computer Modern fonts---for more on
these fonts, see the section ``Obtaining Type 1 Computer Modern.")


\subsection{Making dvips Behave}
If your PostScript output still includes type 3 fonts, you should run dvips with
option ``dvips -Ppdf -G0 -o papername.ps papername.dvi" (If your machine or site
has type 1 fonts, they will probably be loaded.) Note that it is a zero folloing
the ``-G." This tells dvips to use the config.pdf file (and this file refers to a
better font mapping). If that doesn't work, you'll have to download the fonts and
create a font substitution list. 


\subsubsection{dvipdf Script}
Scripts such as dvipdf which ostensibly bypass the Postscript intermediary should not be used since they generally do not instruct dvips to use the config.pdf file.

\subsection{Pdflatex}

Pdflatex is a good alternative solution to the \LaTeX font problem. By using 
the pdftex program instead of straight \LaTeX or \TeX, you will avoid the type
3 font problem altogether. However, there is a problem with Pdflatex that
you might need to overcome. The program often won't embed all
fonts. To solve this potential disaster, you must ensure that all of the fonts are embedded (use
pdffonts).  If they are not, you need to configure pdftex to use a font map file
that specifies that the fonts be embedded.  Also you should ensure that images
are not downsampled or otherwise compressed in a lossy way.

If fonts aren't getting embedded, users should look at the pdftex mailing list for hints on how to configure pdftex or 
pdflatex to properly embed the typefaces: http://tug.org/pipermail/pdftex/2002-July/002803.html 


\subsection{Ghostscript}
LatTeX users using GhsotScript should make sure that they are using v7.04 or newer. The older versions do
not create acceptable PDF files on most platforms.


\subsection{Graphics}
If you are still finding Type 3 fonts in your PDF file, look at your graphics! LaTeX users should check all their imported graphics files as well for font problems!



\subsection{Obtaining Type 1 Computer Modern}

If you {\it must} use Computer Modern for the mathematics in your paper (you can't
use it for the text anymore) you will need to download the type 1 Computer fonts.
They are available without charge from the American Mathematical Society. Point
your browser to the following url to find them:
http://www.ams.org/tex/type1-fonts.html

Type 1 versions of Computer Modern are also available (for free) from the BaKoMa
collection at http://xxx.lanl.gov/ftp/pub/fonts/x-windows/
\subsubsection{Making A Font Substitution List}
Once you've installed the type 1 Computer Modern fonts, you'll need to get dvips to
refrain from embedding the bitmap fonts. To do this, you'll need to create a font
substitution list for use by dvips. Each line of this file should start with the
name of the font that TeX uses, as shown below:

\begin{footnotesize}
\begin{flushleft}
cmb10 $<$/usr/local/lib/tex/fonts/type1/cmb10.pfb \\
cmbsy10 $<$/usr/local/lib/tex/fonts/type1/cmbsy10.pfb \\
cmbsy6 $<$/usr/local/lib/tex/fonts/type1/cmbsy6.pfb \\
cmbsy7 $<$/usr/local/lib/tex/fonts/type1/cmbsy7.pfb \\
cmbsy8 $<$/usr/local/lib/tex/fonts/type1/cmbsy8.pfb \\
cmbsy9 $<$/usr/local/lib/tex/fonts/type1/cmbsy9.pfb \\
cmbx10 $<$/usr/local/lib/tex/fonts/type1/cmbx10.pfb \\
cmbx12 $<$/usr/local/lib/tex/fonts/type1/cmbx12.pfb \\
\end{flushleft}
\end{footnotesize}

In this example, the assumption is that you have PFB versions of the Computer Modern
fonts located in the directory /urs/local/lib/tex/fonts/type1/. The file name
should be the type 1 encoding of the Postscript font in PFB or PFA format.

If your home directory contains a file called .dvipsrc containing the line: ``*	p
+fontMapFileName" that font map will be used by dvips for all the jobs you run. You
can also created a file, like "config.embed" that contains that line. If you do
that, when you invoke dvips with the command ``dvips -P embed ...," dvips will look
for config embed in the current directory (and perhaps your home directory). You
may need to change how dvips looks for config files. To do this, read the
``environment variables" section of the dvips documentation.

If you need more information, or a better and more technical explanation of how to
make this all work, Kendall Whitehouse has written detailed instructions on
"Creating Quality Adobe PDF FIles from TeX with DVIPS." It is available from
Adobe's Web Site, and other sites on the Internet (you'll need to do a quick search
for it). 

\subsection{Checking For Improper Fonts}
Once a PDF has been made, authors should check to ensure that the file contains no Type 3 fonts and further that all 
fonts have been embedded. This step is hardly ever used by authors, and it would save significant time (and money!) 
if they would simply take 45 seconds and do this. This can be done with the pdffonts utility that is included in the 
Xpdf package (http://www.foolabs.com/xpdf/).
 Alternatively, you can use the File--Document Properties--Fonts option in Acrobat Reader; 
if you chose the latter, you should be sure that no other PDF documents are open at the time.

\section{Citations and References}
Be sure to read the ``Formatting Instructions for Authors" paper that was included
in your author kit (and is available on the AAAi website. \LaTeX\ will handle your
citations improperly unless you change its configuration and use the correct
commands. The AAAI style file and the BibTeX files will help you in this regard.
George Ferguson's paper shows you the proper commands necessary to implement
AAAI citation and reference style. 

\section{Illustrations and Figures}
Figures, drawings, tables, and photographs should be placed throughout the paper
near the place where they are first discussed. Do not group them together at the end
of the paper. \LaTeX\ will sometimes put portions of the figure or table in the
margin. If this happens, you need to scale the figure or table down, because {\bf
nothing} (even a line!) is allowed to intrude into the margins.

\section{Electronic Submissions}
To aid in the creation of a permanent electronic archive of all its publications and for creation
of its publications, AAAI requires electronic submission of your paper in PDF format, as well as
its abstract, and author--title information. Please see the Author Formatting Instructions
document for additional information.

As a \LaTeX\ user, you need to pay attention to the special requirements imposed on you. In
particular, although you are required to submit a PDF version of your paper, we may also require
that you submit, in a tar, zipped, or stuffed archive, all the source files (including any
figures) you used to create your paper. If we do request this material, please provide us with a
{\bf single} source file containing your entire paper, including the bibliography. Also please
remove all commented out portions of your paper. We may also request that you send us the figures
that accompany your document. Please do not send us material that is not actually used in the
paper.

If we ask for this material, it is because nearly all the problems we have with electronic
submissions come from persons using \LaTeX\ . Many of the problems would be easy to fix if we had
all the source files, which is why we may ask you to
 send them to us. Even if we can't take the time to fix the file, we can usually tell quite
quickly what is wrong if we have your source, and give you directions on how you might fix it
yourself. 

\section{Possible Bugs in the AAAI Style File}

Some users have found that the aaai.sty does not work properly at their site. They have submitted
suggestions for improvement of the macro. You will find those suggestions in the buglist file
that is part of complete set, and also as a separate file on the AAAI website. Some of these
suggestions have already been implemented, while others seem to be dependent on individual site
conditions. If you're having problems with aaai.sty, we suggest you look at the ``bug list." The
AAAI style is {\bf not} guaranteed to work. It is provided in the hope that it will make the
preparation of papers easier. There are undoubtably bugs in this style. If you make bug fixes,
improvements, etc. please let us know so that we might include them in the buglist.

\section{Inquiries}
If you have any general questions about the preparation or submission of your paper, please
contact AAAI Press. If you have technical questions about implementation of the macros, please
contact an expert at your site. We do not provide technical support for \LaTeX\ or any other
software package. If you are new to \LaTeX\, your paper is fairly straightforward, and doesn't
include multilevel equations, you will probably find that you can format it much faster using a
word-processing program like Microsoft Word. This is especialy true if your paper includes a
number of pictures or graphics.

\section{A Note on Printing}
Some laser printers have a serious problem printing \TeX\ output. These printing devices,
commonly known as ``write-white'' laser printers, tend to make characters too light. To get
around this problem, a darker set of fonts must be created for these devices.

\section{ Acknowledgments}
AAAI is especially grateful to Peter Patel Schneider for his work in implementing the aaai.sty
file, liberally using the ideas of other style hackers, including Barbara Beeton. We also
acknowledge with thanks the work of George Ferguson for his guide to using the style and BibTeX
files --- which has been incorporated into this document and comprises almost all the subsection 
entitled ``Using the AAAI Style File,"  as well as the many others who have, from time to time,
send in suggestions on improvements to the AAAI styles. 


The preparation of the \LaTeX{} and Bib\TeX{} files that implement these instructions was
supported by Schlumberger Palo Alto Research, AT\&T Bell
 Laboratories, Morgan Kaufmann Publishers, and AAAI Press. Bibliography style changes were added
by Sunil Issar. \verb+\+pubnote was added by J. Scott Penberthy. George Ferguson added support for
printing the AAAI copyright slug.


\bigskip
\noindent Thank you for reading these instructions carefully. We look forward to
receiving your camera-ready copy!

\end{document}
